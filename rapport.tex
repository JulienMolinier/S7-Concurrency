\documentclass[12pt,oneside,a4paper]{article}
\usepackage[table]{xcolor}
\usepackage{graphicx}
\usepackage{amsmath}
\usepackage{fancyhdr}
\usepackage{hyperref}

\hypersetup{
    colorlinks=false,
    linktoc=all,
}

\pagestyle{fancy}
\cfoot{\thepage}
\fancyhead{}
\fancyhead[R]{\leftmark}
\renewcommand*\contentsname{Sommaire}

\begin{document}
\title{%
  Rapport \\Gestion de la Concurrence}

\author{LERAS - MOLINIER}
\date{21 décembre 2019}
\maketitle
\newpage

\tableofcontents
\newpage

\section{Diagramme de cas d'utilisation}
\paragraph{}

\newpage
\section{Diagramme de classes}
\paragraph{}

\newpage

\section{Avancée Fonctionnelle et Tests cucumber }

\subsection{Boutique}
\paragraph{}
Le système permet à un manager de changer les horaires d'ouverture de son magasin dans la 
limite ou ce changement n'affecte pas des commandes en attente de récupération
(\textit{ChangeOpenHour.feature}). Coté client celui-ci n'a pas la possibilité de sélectionner un horaire
pour récupérer sa commande hors des horaires d'ouverture du magasin qu'il a choisi. Il peut demander à récupérer
une commande grâce au numéro de commande dans la mesure où il se présente à l'heure qu'il a spécifié ou après
(\textit{PickUpOrder.feature} et \textit{PassOrder.feature}). Le système permet également 
la proposition d'une nouvelle recette chaque mois qui est mise en avant en tant que recette 
du mois (\textit{Cookies.feature}).

\end{document}